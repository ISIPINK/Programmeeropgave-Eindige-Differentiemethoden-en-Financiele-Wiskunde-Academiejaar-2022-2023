\documentclass{article}
\input{preamble.tex}
\usepackage{graphicx}

\begin{document}
\oefening{1}

We gaan de Black-Scholes PDE in het
discretisatie rooster $(s_{i})$ beschouwen
met volgende notatie.
  \begin{align*}
    u',u'_{j}   &= u_{t},u_{t}(s_{j},t)\\
    c^{2},c^{1},c^{0}&= \frac{1}{2} \sigma^{2}s^{2}, rs , r  
  \end{align*}
 Benaderingen van $u$ zullen aangeduid worden met
 $U$.

De Black-Scholes vergelijking met deze notatie wordt:
\[
  u'= c_{2}u_{ss }+ c_{1}u_{s} -c_{0}u
.\]

Dit zijn tweede orde benaderingen voor de afgeleiden van $u$ (tweede orde centraal):

\begin{align*}
  u_{s}  &= \frac{u(s+h)-u(s-h)}{2h} + O(h^{2})\\
  u_{ss} &= \frac{u(s+h)-2u(s)+u(s-h)}{h^{2}} + O(h^{2})
\end{align*}

Hiermee wordt de benaderde Black-Scholes vergelijking
in het rooster:

\[
  U'_{j}= \frac{c_{j}^{2}}{h^{2}}(U_{j+1}-2U_{j}+ U_{j-1})
  + \frac{c_{j}^{1}}{2h}(U_{j+1}-U_{j-1})
  - c_{j}^{0}U_{j} 
.\]
Dit kunnen we eenvoudiger schrijven in matrix vorm.
Voer eerst volgende notatie in 
\[
    D_{j}=
  \left[
  \left(
  \frac{c_{j}^{2}}{h^{2}} + \frac{c_{j}^{1}}{2h}
  \right) ,
  \left(
  -2 \frac{c_{j}^{2}}{h^{2}}- c_{j}^{0}
  \right) ,
  \left(
  \frac{c_{j}^{2}}{h^{2}} - \frac{c_{j}^{1}}{2h} 
  \right) 
  \right]  .\]
Hiermee wordt dit
\[
U'_{j}=D_{j} \left( U_{j-1},U_{j}, U_{j+1} \right)^{T}
 \text{voor } 1 \le j \le m.\]
Nog meer notatie
\begin{align*}
  U' &= (U'_{1}, U'_{2}, ..., U'_{m})^{T}\\
  U_{d} &= (U_{0}, U_{2}, ..., U_{m+1})^{T}\\
  A_{d} &= \text{Diag}(D_{j})\\
  U &= U_{d}\text{ zonder de eerste en laatste rij } \\
  A &= A_{d}\text{ zonder de eerste en laatste kolom} 
\end{align*}
Hiermee wordt dit
\[
  U' = A_{d}U_{d}
.\]

Door de beginvoorwaarden zijn we enkel geïnteresseerd
in $U$. Door de structuur van $A_{d}$ en de 
beginvoorwaarden kunnen we $U_{0}$
en $U_{m+1}$ wegwerken in een term $g(t)$.

\[
  U' = AU + (D_{1})_{0}U_{0}e_{1} + (D_{m})_{2} U_{m+1} e_{m}
.\]

Merk op dat in deze opgave $U_{0}=0$.
\oefening{2}
Kijk naar figuur\ref{fig:mnexp} voor de gevraagde grafiek. Dit suggereert 
dat $\omega = 0$ en $K = 2$ stabiliteit constanten
zijn voor deze semi discretisatie. \\
Als $m \rightarrow \infty$ dan wordt de bijhorende grafiek waarschijnlijk
een rechte.

\begin{figure}
\includegraphics[width=\linewidth]{oefening2.png}
\caption{ $||e^{At}||_{2}$ vs $t$ voor verschillende $m$}\label{fig:mnexp}
\end{figure}

\oefening{3}
Kijk naar figuur\ref{fig:muA} voor $\mu_{2}(A)$ voor verschillende $m$. 
$\mu_{2}(A)$ groeit even snel als $m^{2}$ (kijk grafiek). Dit kan waarschijnlijk
bewezen worden met $\mu_{1}, \mu_{2}$ afschatting. Dit betekent dat voor  
$K$ de stabiliteit constante bij $T = \infty: K > 1$ .

\begin{figure}
\includegraphics[width=\linewidth]{oefening3.png}
\caption{ $(\mu_{2}(A), m^2)$ vs $m$}\label{fig:muA}
\end{figure}

\oefening{4}
In dit geval is de impliciete trapezium regel:
\[
  U_{n+1} = U_{n} + \frac{1}{2} \tau (A U_{n} + g(t_{n}) + A U_{n+1} + g(t_{n+1}))
.\]
Verzamel $U_{n+1}$ (omdat we die niet weten) dit ziet er zo uit:
\[
  (I-\frac{\tau}{2}A)U_{n+1} = U_{n} + \frac{\tau}{2} (AU_{n} + g(t_{n})+ g(t_{n+1}))
.\]
Dit is een ijl stelsel die een unieke oplossing heeft voor kleine $\tau$.
\end{document}

