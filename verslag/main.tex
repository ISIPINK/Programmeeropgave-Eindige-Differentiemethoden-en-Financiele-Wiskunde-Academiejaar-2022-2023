\documentclass{article}
\input{preamble.tex}

\begin{document}
\oefening{1}

We gaan de Black-Scholes PDE in het
discretisatie rooster $(s_{i})$ beschouwen
met volgende notatie.
  \begin{align*}
    u',u'_{j}   &= u_{t},u_{t}(s_{j},t)\\
    c^{2},c^{1},c^{0}&= \frac{1}{2} \sigma^{2}s^{2}, rs , r  
  \end{align*}
 Benaderingen van $u$ zullen aangeduid worden met
 $U$.

De Black-Scholes vergelijking met deze notatie wordt:
\[
  u'= c_{2}u_{ss }+ c_{1}u_{s} -c_{0}u
.\]

Dit zijn tweede orde benaderingen voor de afgeleiden van $u$ (tweede orde centraal):

\begin{align*}
  u_{s}  &= \frac{u(s+h)-u(s-h)}{2h} + O(h^{2})\\
  u_{ss} &= \frac{u(s+h)-2u(s)+u(s-h)}{h^{2}} + O(h^{2})
\end{align*}

Hiermee wordt de benaderde Black-Scholes vergelijking
in het rooster:

\[
  U'_{j}= \frac{c_{j}^{2}}{h^{2}}(U_{j+1}-2U_{j}+ U_{j-1})
  + \frac{c_{j}^{1}}{2h}(U_{j+1}-U_{j-1})
  - c_{j}^{0}U_{j} 
.\]
Dit kunnen we eenvoudiger schrijven in matrix vorm.
Voer eerst volgende notatie in 
\[
    D_{j}=
  \left[
  \left(
  \frac{c_{j}^{2}}{h^{2}} + \frac{c_{j}^{1}}{2h}
  \right) ,
  \left(
  -2 \frac{c_{j}^{2}}{h^{2}}- c_{j}^{0}
  \right) ,
  \left(
  \frac{c_{j}^{2}}{h^{2}} - \frac{c_{j}^{1}}{2h} 
  \right) 
  \right]  .\]
Hiermee wordt dit
\[
U'_{j}=D_{j} \left( U_{j-1},U_{j}, U_{j+1} \right)^{T}
 \text{voor } 1 \le j \le m.\]
Nog meer notatie
\begin{align*}
  U' &= (U'_{1}, U'_{2}, ..., U'_{m})^{T}\\
  U_{d} &= (U_{0}, U_{2}, ..., U_{m+1})^{T}\\
  A_{d} &= \text{Diag}(D_{j})\\
  U &= U_{d}\text{ zonder de eerste en laatste rij } \\
  A &= A_{d}\text{ zonder de eerste en laatste kolom} 
\end{align*}
Hiermee wordt dit
\[
  U' = A_{d}U_{d}
.\]

Door de beginvoorwaarden zijn we enkel geïnteresseerd
in $U$. Door de structuur van $A_{d}$ en de 
beginvoorwaarden kunnen we $U_{0}$
en $U_{m+1}$ wegwerken in een term $g(t)$.

\[
  U' = AU + (D_{0})_{0}U_{0} + (D_{m})_{2} U_{m+1}
.\]
\end{document}

